% Document class (keine Änderungen hier)
\documentclass[a4paper,oneside,bibliography=totoc]{scrartcl}

% Encoding und Sprachen
\usepackage[utf8]{inputenc}
\usepackage[english]{babel}

% Mathematische Pakete
\usepackage{amsmath}
\usepackage{amssymb}
\usepackage{amsfonts}
\usepackage{amsthm} % Hier richtig platziert

% Grafik und Layout
\usepackage{graphicx}
\usepackage{tabularx}
\usepackage{booktabs}
\usepackage{multicol}
\usepackage{subcaption}

% Sonstige nützliche Pakete
\usepackage{latexsym}
\usepackage{csquotes}
\usepackage{cancel}
\usepackage{accents}
\usepackage{makeidx}
\usepackage{listings}
\usepackage{algorithm}
\usepackage{algorithmic}
\renewcommand{\algorithmiccomment}[1]{\hfill\textit{// #1}}

% Hyperlinks und Farben
\usepackage[hidelinks]{hyperref} % `hidelinks` vermeidet sichtbare Link-Kästen
\usepackage[usenames,dvipsnames]{xcolor} % Farbe für Hyperlinks
\hypersetup{
    colorlinks=true,
    citecolor=Green,  % Zitatfarben auf Grün gesetzt
}

% Literaturstil
\usepackage{natbib}
\bibliographystyle{chicagoa}
\setcitestyle{authoryear,round,semicolon,aysep={},yysep={,}}
\let\cite\citep

% Definitionen, Propositionen und Bemerkungen
\newtheorem{defn}{Definition}[section] % Definition-Umgebung, [section] für Nummerierung nach Abschnitten
\newtheorem{prop}{Proposition}[section]
\newtheorem{rem}{Remark}[section]
\begin{document}

\subject{Report - Machine Learning (HWS 2024)} % change to appropriate type
\title{Assignment 2: Logistic Regression}
\author{Arne Huckemann (ahuckema), Elise Wolf (eliwolf)}
\date{\today}
\maketitle


\begin{abstract}
\end{abstract}


\section{Introduction}
\label{ch:intro}



\section{Task 2}

\subsection{Task 2a}

We need to show that if we use a bias term, rescaling (multiplying by a constant), and shifting (adding a constant) features, it leads to Maximum Likelihood (ML) estimates with the same likelihood in the context of logistic regression. Then, we need to explain why we computed z-scores.

Consider a logistic regression model with a bias term:

\[
\text{logit}(P(y=1|x)) = \beta_0 + \beta_1 x_1 + \beta_2 x_2 + \cdots + \beta_p x_p
\]

where \( \text{logit}(P(y=1|x)) = \log\left(\frac{P(y=1|x)}{1 - P(y=1|x)}\right) \).

Let \( x_j' = a_j x_j + b_j \) be the rescaled and shifted version of the feature \( x_j \), where \( a_j \) and \( b_j \) are constants.

The model with the transformed features becomes:

\[
\text{logit}(P(y=1|x')) = \beta_0' + \beta_1' x_1' + \beta_2' x_2' + \cdots + \beta_p' x_p'
\]

Substituting \( x_j' = a_j x_j + b_j \) into the model:

\[
\text{logit}(P(y=1|x')) = \beta_0' + \beta_1' (a_1 x_1 + b_1) + \beta_2' (a_2 x_2 + b_2) + \cdots + \beta_p' (a_p x_p + b_p)
\]

Expanding the terms:

\[
\text{logit}(P(y=1|x')) = \beta_0' + \beta_1' a_1 x_1 + \beta_1' b_1 + \beta_2' a_2 x_2 + \beta_2' b_2 + \cdots + \beta_p' a_p x_p + \beta_p' b_p
\]

Grouping the terms involving \( x_j \):

\[
\text{logit}(P(y=1|x')) = \left( \beta_0' + \sum_{j=1}^p \beta_j' b_j \right) + \sum_{j=1}^p \beta_j' a_j x_j
\]

Let \( \beta_0 = \beta_0' + \sum_{j=1}^p \beta_j' b_j \) and \( \beta_j = \beta_j' a_j \). The model becomes:

\[
\text{logit}(P(y=1|x)) = \beta_0 + \sum_{j=1}^p \beta_j x_j
\]

This shows that the transformed model can be written in the same form as the original model, with appropriately adjusted coefficients. Therefore, the likelihood function remains the same, and the ML estimates are invariant to rescaling and shifting of the features.

\subsection*{Why Compute Z-Scores?}

Computing z-scores standardizes the features to have a mean of 0 and a standard deviation of 1. This is important for several reasons:

\begin{itemize}
    \item \textbf{Numerical Stability}: Standardizing features can improve the numerical stability of optimization algorithms used in ML estimation.
    \item \textbf{Interpretability}: Standardized coefficients are easier to interpret, as they represent the effect of a one standard deviation change in the feature.
    \item \textbf{Comparison}: Standardizing allows for direct comparison of the importance of different features.
\end{itemize}


\section{Conclusions}

\citet{zobel2004} writes:

\blockcquote{zobel2004}{%
  The closing section, or summary, is used to draw together the topics discussed
  in the paper. It should include a concise statement of the paper's important
  results and an explanation of their significance. This is an appropriate place
  to state (or restate) any limitations of the work: shortcomings in the
  experiments, problems that the theory does not address, and so on.

  The conclusions are an appropriate place for a scientist to look beyond the
  current context to other problems that were not addressed, to questions that
  were not answered, to variations that could be explored. They may include
  speculation, such as discussion of possible consequences of the results.

  A \emph{conclusion} is that which concludes, or the end. \emph{Conclusions}
  are the inferences drawn from a collection of information. Write
  "Conclusions", not "Conclusion". If you have no conclusions to draw, write
  "Summary".}

\bibliography{references}


\appendix
\section{Additional Experimental Results}

\citet{zobel2004} writes:

\blockcquote{zobel2004}{%
  Some papers have appendices giving detail of proofs or experimental results,
  and, where appropriate, material such as listings of computer programs. The
  purpose of an appendix is to hold bulky material that would otherwise
  interfere with the narrative flow of the paper, or material that even
  interested readers do not need to refer to. Appendices are rarely necessary.}

In the context of a BSc or MSc thesis, the last sentence often does not hold.

\section{Proof Details}


\clearpage
\section*{Declaration of Honor}

I hereby declare that I have written the enclosed 
project report without the help of third parties and without the use of other
sources and aids other than those listed in the table below
and that I have identified the passages taken from the sources used verbatim or in terms of content as such.
or content taken from the sources used. This work has not been submitted in the same or a similar form
been submitted to any examination authority. I am aware that a false declaration
declaration will have legal consequences.

% Declare below which AI tools you used in the process of writing your work,
% including text, image, code, and data generation. If you used a tool for a
% purpose not included in the list yet, add it to the list.
\begin{center}
  \textbf{Declaration of Used AI Tools} \\[.3em]
  \begin{tabularx}{\textwidth}{lXlc}
    \toprule
    Tool & Purpose & Where? & Useful? \\
    \midrule
    ChatGPT & Rephrasing & Throughout & + \\
    DeepL & Translation & Throughout & + \\
    %ResearchGPT & Summarization of related work & Sec.~\ref{sec:related_work} & - \\
    %Dall-E & Image generation & Figs.~2, 3 & ++ \\
    Github Copilot & Code generation & a01-nb.ipynb & + \\
    %ChatGPT & Related work hallucination & Most of bibliography & ++ \\
    \bottomrule
  \end{tabularx}
\end{center}

\vspace{2cm}
\noindent Signatures\\
\noindent Mannheim, 12. October 2024 \hfill

\end{document}